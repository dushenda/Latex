
\chapter{代码示例}
\label{chap:example}
\section{Euler数学字体示例}
$$
abcdefghijklmnopqrstuvwxyz
$$
$$
ABCDEFGHIJKLMNOPQRSTUVWXYZ
$$
$$
0123456789
$$
\begin{equation}
\begin{split}
\{S_i=0\}=\frac{a_i}{b_i+a_i} \\
\{S_i=1\}=\frac{b_i}{b_i+a_i} \label{1}
\end{split}
\end{equation}

\section{上标引用示例}
\verb|\scite{lshort-cn}|,效果\scite{lshort-cn}。

\section{表格环境加强命令示例}
本模板提供了表格环境下p\{width\}加强版命令,使用L\{width\}等可以在指定宽度的同时指定对齐方式。

\begin{minipage}{\textwidth}
\begin{Codex}[numbers=left]
\begin{table}
\tabcaption{几种命令效果对比的对比}
\label{tab:tblcmp}
\centering
\begin{tabular}{c||l|c|r|p{2.5cm}|L{2.5cm}|C{2.5cm}|R{2.5cm}}
\hline
命令&l&c&r&p\{width\}&L\{width\}&C\{width\}&R\{width\}\\
\hline
效果&左齐&居中&右齐&定宽&左齐定宽&居中定宽&右齐定宽\\
\hline
\end{tabular}
\end{table}
\end{Codex}

\tabcaption{几种命令效果对比的对比}
\label{tab:tblcmp}
\centering
\begin{tabular}{c||l|c|r|p{2.5cm}|L{2.5cm}|C{2.5cm}|R{2.5cm}}
\hline
命令&l&c&r&p\{width\}&L\{width\}&C\{width\}&R\{width\}\\
\hline
效果&左齐&居中&右齐&定宽&左齐定宽&居中定宽&右齐定宽\\
\hline
\end{tabular}
\end{minipage}

\section{自定义代码环境示例}
\label{codex}
\subsection{Code环境}
\begin{minipage}{\textwidth}
\begin{minipage}{0.4\textwidth}
\begin{Code}[label=a.cpp, numbers=left]
This is Code environment
A simple example.
For more options, see fancyvrb's manual.
\end{Code}
\end{minipage}
\hfill\begin{minipage}{0.4\textwidth}
\begin{Verbatim}[fontsize=\scriptsize,baselinestretch=0.9,xleftmargin=3mm,frame=lines,labelposition=all,framesep=5pt]
\begin{Code}[label=a.cpp, numbers=left]
This is Code environment
A simple example.
For more options, see fancyvrb's manual.
\end{Code}
\end{Verbatim}
\end{minipage}
\end{minipage}

\subsection{Codex环境}
\begin{minipage}{\textwidth}
\begin{minipage}{0.4\textwidth}
\begin{Codex}[label=a.cpp, numbers=left]
This is Codex environment
A simple example.
For more options, see fancyvrb's manual.
\end{Codex}
\end{minipage}
\hfill\begin{minipage}{0.4\textwidth}
\begin{Verbatim}[fontsize=\scriptsize,baselinestretch=0.9,xleftmargin=3mm,frame=lines,labelposition=all,framesep=5pt]
\begin{Codex}[label=a.cpp, numbers=left]
This is Codex environment
A simple example.
For more options, see fancyvrb's manual.
\end{Codex}
\end{Verbatim}
\end{minipage}
\end{minipage}

\subsection{CodeScript环境}
\begin{minipage}{\textwidth}
\begin{minipage}{0.4\textwidth}
\begin{CodeScript}[label=a.cpp, numbers=left]
This is CodeScript environment
A simple example.
For more options, see fancyvrb's manual.
\end{CodeScript}
\end{minipage}
\hfill\begin{minipage}{0.4\textwidth}
\begin{Verbatim}[fontsize=\scriptsize,baselinestretch=0.9,xleftmargin=3mm,frame=lines,labelposition=all,framesep=5pt]
\begin{CodeScript}[label=a.cpp, numbers=left]
This is CodeScript environment
A simple example.
For more options, see fancyvrb's manual.
\end{CodeScript}
\end{Verbatim}
\end{minipage}
\end{minipage}

\subsection{CodexScript环境}
\begin{minipage}{\textwidth}
\begin{minipage}{0.4\textwidth}
\begin{CodexScript}[label=a.cpp, numbers=left]
This is CodexScript environment
A simple example.
For more options, see fancyvrb's manual.
\end{CodexScript}
\end{minipage}
\hfill\begin{minipage}{0.4\textwidth}
\begin{Verbatim}[fontsize=\scriptsize,baselinestretch=0.9,xleftmargin=3mm,frame=lines,labelposition=all,framesep=5pt]
\begin{CodexScript}[label=a.cpp, numbers=left]
This is CodexScript environment
A simple example.
For more options, see fancyvrb's manual.
\end{CodexScript}
\end{Verbatim}
\end{minipage}
\end{minipage}

\section{表格示例}
具体代码请参考源文件./chapter/chap-example.tex。
\begin{table}[htbp]
\centering
\caption{基于因子分析的失配补偿结果}
\label{tab:jfa-gmm-ubm}
\begin{tabular}{cccccc}
    \toprule
    &\multirow{2}{*}{\#Mix}&\multicolumn{2}{c}{No-norm}
    &\multicolumn{2}{c}{Tnorm}\\
    \cline{3-4} \cline{5-6}
		&		& EER(\%) 	& MinDCF & EER(\%) 	& MinDCF\\
    \midrule
	\multirow{3}{*}{GMM-UBM}
    &256 		& 12.43 	& 0.0647	& 12.85    & 0.0580\\
    &512 		& 10.02 	& 0.0464	& 8.88 	   & 0.0370\\
    &1024 		& 9.97 	    & 0.0457	& 8.72 	   & 0.0372\\
    \midrule
	\multirow{3}{*}{Factor Analysis}
    &256 		& 8.09 	& 0.0331 	& 7.39 	& 0.0319\\
    &512 		& 7.08 	& 0.0305 	& 6.53 	& 0.0292\\
    &1024 		& 6.83 	& 0.0295 	& \textbf{6.29} 	& \textbf{0.0279}\\
 \bottomrule
\end{tabular}
\end{table}

\section{算法示例}
具体代码请参考源文件./chapter/chap-example.tex。
\IncMargin{1em}
\begin{algorithm}
\SetKwData{Left}{left}\SetKwData{This}{this}\SetKwData{Up}{up}
\SetKwFunction{Union}{Union}\SetKwFunction{FindCompress}{FindCompress}
\SetKwInOut{Input}{input}\SetKwInOut{Output}{output}
\Input{$O_t,UBM,U$}
\Output{$x,y$}
\BlankLine
\emph{$y\leftarrow 0;$$x_h\leftarrow 0;$$h=1,...,H$ }\;
\For{$i=1$ \KwTo Number of E-M iterations}{
\emph{E Step}:\\
\For{$h=1$ \KwTo $H$}{\label{forins}
对于每一条语音段,计算其EM统计量(零阶统计量$N_h$,一阶统计量$S_{X,h}$\;
}
计算每一个人所有语音段的零阶统计量$N$\\
计算每一个人所有语音段的一阶统计量$S$\\
\emph{M Step}:\\
\For{$j=1$ \KwTo Number of Gauss-Seidel iterations}{
\For{$h=1$ \KwTo $H$}{\label{forins}
估计每一语音段$h$的失配因子$x_h$
}
估计模型的话者因子$y$
}
}
\Return{$\mu = m+Dy$}
\caption{disjoint decomposition}\label{algo_disjdecomp}
\end{algorithm}\DecMargin{1em}

\section{引用参考文献示例}
参考文献测试:\citep{deng:01a}