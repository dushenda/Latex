%-*-coding:
%presentation.tex
%英语展示
\documentclass[UTF8]{ctexart}

\usepackage{geometry}
\geometry{a4paper,left=3cm,right=3cm,top=1cm,bottom=1cm}

\title{英语展示}
\author{第一组}
\date{\today}

\begin{document}
	\maketitle
	\section{介绍}
	社会心理学中,对他人“印象”的形成有一个非常重要的规律:首因效应(primacy
	effect),即“第一印象效应”,是说人们比较重视最先得到的信息,并据此对他人作出判断。

	Luchins 在1957年做了这样一个实验。他选择了一个小故事作自变量。故事描述了一个名叫吉姆的小学生的生活片断,由上下两段组成,分别把吉姆描写成为一个热情外向的孩子和一个冷淡内向的孩子,让我们来听听这两个故事。
	\section{故事一}
	吉姆离家去买文具。他和两个朋友一起走在洒满阳光的街道上,边走边晒太阳。吉姆走进文具店,店里挤满了人。他一面等待售货员招呼他,一面和熟人聊天。买好文具向外走的时候又遇到了熟人,他就停下来和同学打招呼。后来告别了朋友往学校走,路上遇到了一个前天晚上才刚刚认识的女孩子,他们说了几句话之后就分手了。吉姆来到了学校。
	\section{故事二}
	放学后,吉姆独自一人离开教室,走出了学校。他开始延着漫长的路步行回家。街道上的阳光非常耀眼,于是吉姆走到街道阴凉的一边。迎面而来的街道渐渐消失在他身后,他看到一位前一天晚上遇到过的那个漂亮的女孩。吉姆走进了一家糖果店,店里挤满了学生,他看到一些熟悉的面孔。吉姆静静的等待着,直到引起柜台服务员的注意之后才买到了饮料。他坐在一张靠墙的椅子上喝饮料。喝完之后他就回家了。
	\section{总结}
	结果发现, 先呈现上半段再呈现下半段,有78\%的被试认为吉姆是外向的;先呈现后半段再呈现上半段则只有18\%的被试认为吉姆是外向的,63\%认为其内向。结果说明了首先接受到的信息直接影响个体对他人的认识,即首因效应。
	
	其原理有以下解释:一是人们容易忽略后面的信息,因为一旦人们觉得自己有足够的信息来作判断,就不再或很少注意随后的信息;二是在知觉者的心目中,后面的信息常常不如开始的信息重要、有价值。
	
\end{document}